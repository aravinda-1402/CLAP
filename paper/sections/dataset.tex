The dataset is generated by \texttt{clap build-data} (or as part of \texttt{clap run}) with a fixed seed for reproducibility. It comprises:

\begin{itemize}
  \item \textbf{Base cases} (\texttt{cases\_base.jsonl}): Each record has \texttt{base\_id}, \texttt{domain}, synthetic \texttt{demographics} (age group, sex), \texttt{comorbidities}, \texttt{meds}, \texttt{vitals}, \texttt{labs}, \texttt{allergies}, \texttt{pregnancy\_flag}, \texttt{notes}, and a natural-language \texttt{summary}. Internal consistency is enforced (e.g., renal impairment affects labs; pregnancy only for applicable demographics).
  \item \textbf{Family variants} (\texttt{cases\_family.jsonl}): Each has \texttt{variant\_id}, \texttt{base\_id}, \texttt{variant\_type} (renal\_impairment, pregnancy\_toggle, allergy\_key\_med, interaction\_introduced), \texttt{expected\_change\_spec} (machine-readable expected/forbidden changes), and \texttt{summary}.
  \item \textbf{Suites}: \texttt{suites/nrt100.jsonl} (100 must-not-miss safety cases with \texttt{expected\_risk\_flags}), \texttt{suites/ambiguity.jsonl}, \texttt{suites/policy\_conflict.jsonl}.
\end{itemize}

JSON schemas in \texttt{data/schema/} validate base cases, family variants, suite entries, model output, and the audit packet. No PHI is used; all content is explicitly synthetic.
